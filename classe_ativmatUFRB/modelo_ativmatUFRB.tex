% arara: {options: ['--output-directory=docs_saida']}
% arara: {options: ['--output-directory=docs_saida']}

%
\documentclass{cls/ativmatUFRB}
%
%=======================================
% Informações do Título da Lista
%=======================================
\tituloDaLista{Título da Lista} %------> Coloque aqui o título da sua lista.
\prof{Fulado Cicrano de Tal} %---------> Nome do Professor.
\disciplina{Miscelânea} %--------------> Disciplina ministrada (Cálculo I, Funções de uma Variável Complexa, etc).
\curso{Licenciatura em Matemática} %---> Curso onde ministra a disciplina.
\semestre{8} %-------------------------> Semestre que onde está inserida a DISCIPLINA.
\numeroDaLista{I} %--------------------> Número da Lisda de Atividade (colocar números em algarismos romanos).
%=======================================

%==============================================================================
% Início da Lista de Atividade
%==============================================================================
\begin{document}
%                                          * O comando \titulo gera o cabeçalho 
%                                            estilizado (com logo da UFRB).
%                                          * Não apagar esse comando!
%
  \titulo 
% 
\begin{atividade} %---------------------------------------------------------->>
%
%                                   * O comando \topico{} deixa o nome da seção
%                                     sombreado com cinza.
%                                   * Os comando \subtopico{} e \subsubtopico{}
%                                     diminuem as fontes, mas não coloca a faixa
%                                     cinza
% Tópico 01 -------------------------------------------------------------------
\topico{Sequências \& Séries} 
  %==============================================================================
% Questão 01
%------------------------------------------------------------------------------
%
% Note que as questões são iniciadas com \questao. ¯\_(ツ)_/¯
%
\questao Na Figura~\ref{fig:espiral} temos uma espiral\footnote{%
  \urlb{https://en.wikipedia.org/wiki/Spiral} %--------------------------------> \urlb{} deixa a url em negrito
                                              %                                  (caso contrário use \url{})
}
formada por semicírculos cujos centros pertencem ao eixos das abscissas. 
Se o raio do primeiro semicírculo é igual a $ 1 $ e o raio de cada semicírculo é 
igual a metade do semicírculo anterior, determine:

\begin{figure}[!htbp] %---------------------------------------------------------> ambiente para figuras.
  \centering %------------------------------------------------------------------> centralizando a figura.
  \includegraphics[width = 0.35 \linewidth]{espiral} %--------------------------> comando para inserir figura.
  \caption{Um tipo de espiral} %------------------------------------------------> legenda da figura.
  \label{fig:espiral} %---------------------------------------------------------> marca p/ referências cruzadas
\end{figure}

\begin{itens} %-----------------------------------------------------------------> ambiente para listas.
  \item o comprimento da espiral. \Resp{$2\pi$} %-------------------------------> comando para as respostas.
  \item a abscissa do ponto assintótico da espiral. \Resp{$4/3$}
\end{itens}
%==============================================================================
 %-------------------------------------> Questão 01
  %==============================================================================
% Questão 02 
%------------------------------------------------------------------------------
\questao Uma sequência $ (x_n)_{_{n\in\mathbb{N}}} $ é dita \textit{Sequência de Cauchy} 
quando:
\begin{center}
  \fbox
  {
    \begin{minipage}{.7\textwidth}
      Dado arbitrariamente um número real $ \displaystyle \varepsilon > 0 $, 
      pode-se obter $ \displaystyle n_{0}\in \mathbb{N} $ tal que:
      \[
        m > n_{0} \hspace{0.5em} \textrm{ e } \hspace{0.5em} n > n_{0} 
        \Longrightarrow
        \left|x_{m} - x_{n}\right| < \varepsilon.
      \]
    \end{minipage}
  }
\end{center}
\begin{itens}
  \item Mostre que toda sequência convergente é de Cauchy.
\end{itens}

%============================================================================== %-------------------------------------> Questão 02
  %==============================================================================
% Questão 03
%------------------------------------------------------------------------------

\questao Classifique as afirmações abaixo com \textbf{V}~(Verdadeiro) ou 
\textbf{F}~(Falso). Justificando cada uma. Procure justificar as afirmações 
falsas com um contra exemplo.
\begin{multicols}{2} %---------------------------------------------------------> múltiplas colunas (duas cols).
  \begin{itens}[\vf] %---------------------------------------------------------> \vf produz ( ).
    \item toda sequência decrescente limitada é convergente e seu limite é zero.
    \item toda sequência divergente é não limitada.
    \item toda sequência alternada é divergente.
    \item toda sequência convergente é limitada.
    \item toda sequência limitada é convergente.
    \item toda sequência limitada é monótona.
    \item toda sequência monótona é convergente.
    \item toda sequência divergente é não monótona. 
  \end{itens}
\end{multicols}

%==============================================================================
 %-------------------------------------> Questão 03
%
% Tópico 02 -------------------------------------------------------------------
\topico{Cálculo Vetorial e Integral} 
  \subtopico{Integrais Triplas} 
  \subsubtopico{Coordenadas Cilíndricas e Esféricas}
    %==============================================================================
% Questão 04
%------------------------------------------------------------------------------
\questao Use coordenadas esféricas e calcule as seguintes integrais:

\begin{itens}
 \item 
  $
   \displaystyle
   \int_{-2}^{2}\!
   \int_{-\sqrt{4 - x^2}}^{\sqrt{4 - x^2}}\!
   \int_{\sqrt{x^2 + y^2}}^{\sqrt{8 - x^2 - y^2}}{\left(x^2 + y^2 + z^2\right)\dd z\dd y\dd x}
  $. 
 \Resp{$ \frac{256\pi}{5}\left(\sqrt{2} - 1/2\right) $}
 \item 
  $
   \displaystyle
   \int_{0}^{\sqrt{2}}\!\!
   \int_{y}^{\sqrt{4 - y^2}}\!\!
   \int_{0}^{\sqrt{4 - x^2 - y^2}}{\sqrt{x^2 + y^2 + z^2}\dd z\dd x\dd y}
  $. 
 \Resp{$ \pi $}
\end{itens}
%============================================================================== %-----------------------------------> Questão 04
  \subtopico{Integrais de Linha}
    %==============================================================================
% Questão 05
%------------------------------------------------------------------------------
\questao Seja $\Gamma$ o segmento de reta que liga a origem ao ponto 
$\textrm{A} = (1,1,1)$. 
Calcule $\displaystyle \int_{\Gamma}\vv{F}\dd{\Gamma}$, onde:
\[
 \vv{F}(x,y,z) = xy\versor{i} - y\versor{j} + 1\versor{k}.
\]
%==============================================================================
 %-----------------------------------> Questão 05
%
% Tópico 03 -------------------------------------------------------------------
\topico{Álgebra Linear}
  \subtopico{Sistemas Lineares}
    %==============================================================================
% Questão 06
%------------------------------------------------------------------------------

\questao (Fuvest-SP-Adap.) Considerando o sistema 
\begin{center}
 \systeme{x+2y+3z=14,4y+5z=23,6z=18}, %---------------------------> Comando para construção de sistemas de equações. 
\end{center}
então o valor de $x$ é igual a:
\altercols{5}{-2}{0}{1}{3}{27} %------------------------------------------------> Comando para múltipla escolha em 5 (cinco colunas).
%==============================================================================
 %-----------------------------------> Questão 06
    %==============================================================================
% Questão 07
%------------------------------------------------------------------------------
\questao  (IBMEC) Num prédio existem 12 andares, todos ocupados. 
Alguns, por 4 pessoas, outros, por apenas 2 pessoas, num total de 38 pessoas. 
O número de andares ocupados por 2 pessoas é:
\duascolunas{4}{5}{6}{8}{19} %--------------------------------------------------> Comando para múltipla escolha em 2 (duas colunas) fixas.
%==============================================================================
 %-----------------------------------> Questão 07
%
% Tópico 04 -------------------------------------------------------------------
\topico{Estatística}
  %==============================================================================
% Questão 08
%------------------------------------------------------------------------------

\questao Uma pesquisa realizada sobre a preferência dos consumidores por três 
categorias de veículos $ A $, $ B $ e $ C $ de uma indústria automobilística 
revelou que dos 500 entrevistados:

\begin{multicols}{2}
  \begin{itens}[I)]
    \item 210 preferiam o veículo $ A $.
    \item 230 preferiam o veículo $ B $.
    \item 160 preferiam o veículo $ C $.
    \item 90 preferiam o v eículo $A$~e~$B$.
    \item 90 preferiam os veículos $A$~e~$C$.
    \item 70 preferiam os veículos $B$~e~$C$.
  \end{itens}
\end{multicols}

Um consumidor é selecionado ao acaso entre os entrevistados.
Calcule a probabilidade de que:

\begin{itens}
  \item Ele prefira as três categorias.
  \item Ele prefira somente uma das categorias.
  \item Ele prefira apenas a categoria $ A $.
\end{itens}
%==============================================================================
 %-------------------------------------> Questão 08
  %==============================================================================
% Questão 09
%------------------------------------------------------------------------------

\questao Cinco corredores foram examinados para determinar a quantidade máxima 
de aspiração de oxigênio, que é uma medida usada para caracterizar a situação 
cardiovascular de uma pessoa.
Os resultados estão na Tabela~\ref{tab:corredor}, onde ``$x$'' é o número de 
segundos no melhor tempo feito em um quilômetro e ``$y$'' é o número de 
mililitros por minuto, por quilograma de peso corporal da aspiração máxima de 
oxigênio do corredor.

\begin{table}[!htbp] %----------------------------------------------------------> Ambiente para Tabela/ Posicionamento no texto.
 \caption{Segundos por melhor corredor} %---------------------------------------> Legendas da tabela.
 \label{tab:corredor} %---------------------------------------------------------> Marcação para referencias cruzadas.
 \centering %-------------------------------------------------------------------> Centraliza a tabela.
 \begin{tabular}{l c c c c c} %---------------------> Tabela estilizada/ Cada letra é uma coluna (c: centralizada; l: à esquerda; r: à direita).
  \toprule %------> Linha superior.
                & \textbf{Corredor A} & \textbf{Corredor B} & \textbf{Corredor C} & \textbf{Corredor D} & \textbf{Corredor E}\\
  \midrule %------> Linha média.
   $\mathbf{x}$ & 300,5               & 350,6               & 407,3               & 326,2               & 512,8\\
   $\mathbf{y}$ & 350,2               & 325,8               & 375,6               & 418,5               & 400,2\\
  \bottomrule %---> Linha inferior.
 \end{tabular}
\end{table}
	
\begin{itens}
  \item Trace o diagrama de dispersão.
  \item Ache a reta de regressão para os dados da tabela.
  \item Use a reta de regressão para estimar a máxima aspiração de oxigênio de 
        um corredor, cujo melhor tempo em uma milha é de \unit{340,4.s}. %------> `\unit{}` padroniza unid. de media. Note o 'ponto' para separação.
\end{itens}
%============================================================================== %-------------------------------------> Questão 09
%
% Tópico 05 -------------------------------------------------------------------
\topico{Variáveis Complexas}
  %==============================================================================
% Questão 10
%------------------------------------------------------------------------------
\questao (\textsf{UFMS-adap.}) Sobre o número complexo $ z $ que satisfaz a equação 
\[
  2 \bar{z} + iz + 1 - i = 0,
\]
julgue os itens abaixo em \textbf{V}~(verdadeiro) ou \textbf{F}~(falso).
\begin{multicols}{3}
 \begin{itens}[\vf]
   \item $ \left| z \right| = \sqrt{z} $.
   \item $ \textrm{Re} \left( z \right) + \textrm{Im} \left( z \right) = 0 $.
   \item $ \bar{z} = -1 + i $.
   \item $ z $ é um número real.
   \item $ z^{2} = i $.
 \end{itens}
\end{multicols}
%==============================================================================
 %-------------------------------------> Questão 10
  %==============================================================================
% Questão 11
%------------------------------------------------------------------------------

\questao Sendo $ \varphi\colon [0,2\pi]\to\mathbb{C} $, definida por 
$ \varphi(t) = 1 + e^{it} $, tal que $ \Phi = \varphi\left([0,2\pi]\right) $, 
encontre:
\[
  \oint_{\Phi}\frac{1}{z^2-1} \dd{z} %------------------------------------------> Comando para integral circular com orientação ``positiva''.
\]
de duas formas:
\begin{itens}
  \item Diretamente (usando parametrização).
  \item Usando o \texttt{Teorema da Integral de Cauchy}.
\end{itens} 
%==============================================================================
 %-------------------------------------> Questão 11
%
\end{atividade} % <<-----------------------------------------------------------
%
%------------------------------------------------------------------------------
\end{document}
%==============================================================================