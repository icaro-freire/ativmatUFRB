%==============================================================================
\section{Antes de começar\ldots}
%==============================================================================
%
%------------------------------------------------------------------------------
\subsection{Sobre as Versões}
%------------------------------------------------------------------------------
%
\subsubsection{Versão anterior} %----------------------------------------------
%
Para informações sobre a versão \texttt{v1.6} acesse: 
\begin{center}
  \href{https://github.com/icaro-freire}{https://github.com/icaro-freire} 
  \marginpar{\qrcode[height = 1cm]{https://github.com/icaro-freire}}
\end{center}

%
\subsubsection{Versão atual} %-------------------------------------------------
%
A versão atual\footnote{A ideia é fazer a versão convergir ao número de ouro:
$\varphi = 1,61803 39887 49894 84820\ldots$},
\texttt{v1.61}, da classe \texttt{ativmatUFRB.cls} foi concluída em 16/11/2020.

As mudanças realizadas, listadas abaixo, visam implementar aspectos relevantes 
no código, mas não modificam substancialmente o aspecto geral que a classe 
propõe-se a oferecer.

\begin{itemize}
 \item \textbf{Boas Práticas.} Todo o código foi revisto e organizado para que 
  outros professores ou alunos possam modificá-lo mais prazerosamente; além 
  disso, houve uma atualização nos nomes dos comandos para adequação de boas 
  práticas em linguagens de marcação (ou programação): iniciar comandos com 
  letras minúsculas, por exemplo.
  
  \begin{table}[!h]
   \centering
   \begin{tabular}{lcccl}
    \toprule
     \multicolumn{1}{c}{\textbf{v1.6}} &&&& \multicolumn{1}{c}{\textbf{v1.61}}\\
    \midrule
     \verb|\TituloDaLista{}|           &&&& \verb|\tituloDaLista{}| \\
     \verb|\Prof{}|                    &&&& \verb|\prof{}| \\
     \verb|\Disciplina{}|              &&&& \verb|\disciplina{}| \\
     \verb|\Curso{}|                   &&&& \verb|\curso{}| \\
     \verb|\Semestre{}|                &&&& \verb|\semestre{}|\\
     \verb|\NumeroDaLista{}|           &&&& \verb|\numeroDaLista{}| \\
    \bottomrule
   \end{tabular}
  \end{table}
  
 \item \textbf{Pacotes.} O pacote \texttt{inputenc} foi retirado, pois desde 
  2018 o \textit{kernel} do \LaTeX\ já o carrega por padrão, caso sua
  codificação seja a \textsc{utf-8} (a qual é recomendada, ou seja, salve seus
  arquivos \texttt{.tex} com codificação \texttt{utf-8}).
  Também foi retirado o pacote \texttt{amsfonts}, pois o mesmo já está incluso
  no pacote \texttt{amssymb}.
  Outros pacotes foram acrescentados à classe, como você pode verificar na 
  Seção~\ref{pacotes}.
  
 \item \textbf{Comandos.} Foram criados dois comandos para sanar uma dificuldade
  quando altera-se a compilação, aqui considerada apenas com as seguintes 
  possibilidades: \texttt{LuaLaTeX}, \texttt{XeLaTeX} ou \texttt{pdfLaTeX}.
  Então, caso a compilação seja com algum dos dois primeiros (o que é recomendado 
  atualmente), o comando \verb|\usandoXeLuaLaTeX| deve ser inserido no preâmbulo
  do seu documento.
  Entretanto, se a compilação for (ainda) com \texttt{pdfLaTeX}, deve ser inserido
  o comando \verb|\usandopdfLaTeX|.
  
 \item \textbf{Cabeçalho.} Foi feita uma leve melhora na harmonização das caixas
  ovais que o compõe, bem como em espaçamentos e tamanho da fonte em nichos do
  mesmo.
\end{itemize}
%
%------------------------------------------------------------------------------
\subsection{Projeto de Extensão \LaTeX\ CFP}
%------------------------------------------------------------------------------
%
A motivação para desenvolver esta classe vem do \emph{Projeto de Extensão}, 
cadastrado no 
\href{https://www.ufrb.edu.br/cfp/}{Centro de Formação de Professores},
\marginpar{\qrcode[height = 1cm]{https://www.ufrb.edu.br/cfp/}}
intitulado: \emph{\LaTeX\ para o Professor de Matemática}.
Tal projeto é ofertado (parcialmente) em forma de curso, que versa sobre a 
confecção de materiais didáticos impressos (e também visuais, como
apresentações) com alta qualidade tipográfica usando o programa \LaTeX; bem 
como, no desenvolvimento de classes extra-oficiais (lista de atividade, 
avaliações, trabalho de conclusão de curso, etc.) para o curso de Licenciatura
ou Bacharelado em Matemática da UFRB.