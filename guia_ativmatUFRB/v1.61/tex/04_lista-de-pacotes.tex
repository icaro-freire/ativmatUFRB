%==============================================================================
\section{Lista de pacotes já instalados}\label{pacotes}
%==============================================================================

A classe \texttt{ativmatUFRB.cls} já contém alguns pacotes básicos para 
construção de uma lista de atividade em matemática.
Listamos todos os pacotes dessa classe abaixo.
Caso você precise de algum outro pacote, deve inseri-lo no preâmbulo do arquivo
\texttt{.tex}.

%------------------------------------------------------------------------------
\subsection*{\texttt{\textbackslash usepackage\{ifluatex, ifxetex\}}} \marginpar{\qrcode[height = 1cm]{https://www.ctan.org/pkg/ifluatex}}
%------------------------------------------------------------------------------
Esses dois pacotes criam condicionais para que o \LaTeX\ reconheça quais os 
pacotes devem ser compilados ou com \texttt{XeLaTeX}, ou com \texttt{LuaLaTeX},
ou com \texttt{pdfLaTeX}.\\
\texttt{ifluatex}: { \small \url{https://www.ctan.org/pkg/ifluatex} }\\
\texttt{ifxetex}: { \small \url{https://ctan.dcc.uchile.cl/macros/generic/iftex/iftex.pdf} } \marginpar{\qrcode[height = 1cm]{https://ctan.dcc.uchile.cl/macros/generic/iftex/iftex.pdf}}

%------------------------------------------------------------------------------
\subsection*{\texttt{\textbackslash usepackage\{fontspec\}}} 
%------------------------------------------------------------------------------
Serve, dentre outras coisas, para escolher fontes \textit{OpenType}.
Foi desenvolvido para uso com compiladores mais modernos como LuaLaTeX ou 
XeLaTeX.\\
{\small \url{https://ctan.dcc.uchile.cl/macros/unicodetex/latex/fontspec/fontspec.pdf}}
\marginpar{\qrcode[height = 1cm]{https://ctan.dcc.uchile.cl/macros/unicodetex/latex/fontspec/fontspec.pdf}}

%------------------------------------------------------------------------------
\subsection*{\texttt{\textbackslash usepackage\{polyglossia\}}}
%------------------------------------------------------------------------------
É um substituto do pacote \texttt{babel} (veja mais abaixo) para usuários de 
XeLaTeX ou LuaLaTeX.\\
{\small \url{https://ctan.dcc.uchile.cl/macros/unicodetex/latex/polyglossia/polyglossia.pdf}}
\marginpar{\qrcode[height = 1cm]{https://ctan.dcc.uchile.cl/macros/unicodetex/latex/polyglossia/polyglossia.pdf}}

%------------------------------------------------------------------------------
\subsection*{\texttt{\textbackslash usepackage[T1]\{fontenc\}}}
%------------------------------------------------------------------------------
Pacote para codificação de saída de fonte.
Com ele, ocorre corretamente a hifenação das diversas fontes possivelmente 
usadas no texto.\\
{\small\url{https://www.ctan.org/pkg/fontenc}}
\marginpar{\qrcode[height = 1cm]{https://www.ctan.org/pkg/fontenc}}

%------------------------------------------------------------------------------
\subsection*{\texttt{\textbackslash usepackage\{emerald\}}}
%------------------------------------------------------------------------------
\emph{Emerald} é um pacote que dá suporte a algumas fontes gratuitas ECF 
(\emph{Emerald City Fontwerks}) no \LaTeX.
Ela foi usada para produzir, de forma estilizada, o nome 
{\ECFIntimacy Lista de Atividade \ldots} no cabeçalho.\\
{\small\url{http://linorg.usp.br/CTAN/fonts/emerald/doc/emerald.pdf}}
\marginpar{\qrcode[height = 1cm]{http://linorg.usp.br/CTAN/fonts/emerald/doc/emerald.pdf}}

%------------------------------------------------------------------------------
\subsection*{\texttt{\textbackslash usepackage\{lmodern\}}}
%------------------------------------------------------------------------------
Uma família de fontes Latim Modern que produz o ``corpo de texto'' da classe 
\texttt{ativmatUFRB.cls}.\\
{\small\url{https://www.ctan.org/tex-archive/fonts/lm/}}
\marginpar{\qrcode[height = 1cm]{https://www.ctan.org/tex-archive/fonts/lm/}}

%------------------------------------------------------------------------------
\subsection*{\texttt{\textbackslash usepackage[brazilian]\{babel\}}}
%------------------------------------------------------------------------------
Este pacote gerencia regras tipográficas (e outras) culturalmente determinadas 
para uma ampla variedade de idiomas.
Por exemplo, traduz elementos internos de diversas classes \textit{standart} do
\LaTeX.\\
{\small\url{http://linorg.usp.br/CTAN/macros/latex/required/babel/base/babel.pdf}}
\marginpar{\qrcode[height = 1cm]{http://linorg.usp.br/CTAN/macros/latex/required/babel/base/babel.pdf}}

%------------------------------------------------------------------------------
\subsection*{\texttt{\textbackslash usepackage[explicit]\{titlesec\}}}
%------------------------------------------------------------------------------
Esse pacote fornece uma interface para modificar/criar comandos para seleção de 
vários estilos de título.
Veja a documentação para mais detalhes:\\ 
{\small\url{http://linorg.usp.br/CTAN/macros/latex/contrib/titlesec/titlesec.pdf}}
\marginpar{\qrcode[height = 1cm]{http://linorg.usp.br/CTAN/macros/latex/contrib/titlesec/titlesec.pdf}}

%------------------------------------------------------------------------------
%\subsection*{\texttt{\textbackslash usepackage[utf8]\{inputenc\}}}
%------------------------------------------------------------------------------
%Pacote para codificação de entrada. Dentre outras coisas, permite a acentuação diretamente pelos caracteres do nosso teclado. \emph{É obrigatório que seu arquivo \texttt{.tex} esteja salvo em codificação \textsc{utf8}}.\\
%{\small\url{http://linorg.usp.br/CTAN/macros/latex/base/inputenc.pdf}}

%------------------------------------------------------------------------------
\subsection*{\texttt{\textbackslash usepackage\{geometry\}}}
%------------------------------------------------------------------------------
O pacote fornece uma interface de usuário fácil e flexível para personalizar o 
layout da página.
No caso da nossa classe, usamos a saída para papel A4, com margem superior de
1cm e as outras com 1.5cm.\\
{\small\url{http://linorg.usp.br/CTAN/macros/latex/contrib/geometry/geometry.pdf}}
\marginpar{\qrcode[height = 1cm]{http://linorg.usp.br/CTAN/macros/latex/contrib/geometry/geometry.pdf}}

%------------------------------------------------------------------------------
\subsection*{\texttt{\textbackslash usepackage\{lipsum\}}}
%------------------------------------------------------------------------------
Este pacote oferece acesso fácil ao texto fictício do \emph{Lorem Ipsum}, 
separando-o em parágrafos.
Serve para testes tipográficos com palavras que simulam uma escrita mais 
adequada do que repetições de palavras (como ``texto, texto, texto'').\\
{\small\url{http://linorg.usp.br/CTAN/macros/latex/contrib/lipsum/lipsum.pdf}}
\marginpar{\qrcode[height = 1cm]{http://linorg.usp.br/CTAN/macros/latex/contrib/lipsum/lipsum.pdf}}

%------------------------------------------------------------------------------
\subsection*{\texttt{\textbackslash usepackage\{amsmath,amsthm,amssymb,amscd\}}}
%------------------------------------------------------------------------------
Diversos pacotes da Sociedade Americana de Matemática (\emph{American 
Mathematical Society}).
Os pacotes \AmS-\TeX\ são fundamentais para escrita matemática.
Segue a documentação de cada um deles:
\begin{description}
	 \item[amsmath] Para diversos comandos de escrita matemática.\\
   {\small\url{http://linorg.usp.br/CTAN/macros/latex/required/amsmath/amsldoc.pdf}}
   \marginpar{\qrcode[height = 1cm]{http://linorg.usp.br/CTAN/macros/latex/required/amsmath/amsldoc.pdf}}
	 \item[amsthm]  Para configurações de teoremas, corolários, etc.\\
   {\small\url{http://linorg.usp.br/CTAN/macros/latex/required/amscls/doc/amsthdoc.pdf}}
   \marginpar{\qrcode[height = 1cm]{http://linorg.usp.br/CTAN/macros/latex/required/amscls/doc/amsthdoc.pdf}}
	 %\item[amsfonts] Diversos símbolos e fontes\footnote{veja uma tabelas mais 
  % resumida:
  % \url{http://milde.users.sourceforge.net/LUCR/Math/mathpackages/amsfonts-symbols.pdf}}.
  % Por exemplo, produz a simbologia características de conjuntos numéricos 
  % ($\mathbb{N}$ é produzido com o comando \verb|$\mathbb{N}$|).\\
  % {\small\url{http://linorg.usp.br/CTAN/fonts/amsfonts/doc/amsfndoc.pdf}}
	 \item[amssymb] Diversos símbolos matemáticos\footnote{veja esse \emph{link} 
   para visualização mais rápida: 
   \url{https://www.rpi.edu/dept/arc/training/latex/LaTeX_symbols.pdf}}.
   Nesse pacote já está incluso o pacote \texttt{amsfonts}, que pode produzir
   a simbologia características de conjuntos numéricos ($\mathbb{N}$.
   Para tanto,  basta usar o comando \verb|$\mathbb{N}$|)\\
   {\small\url{http://texdoc.net/texmf-dist/doc/fonts/amsfonts/amssymb.pdf}}
   \marginpar{\qrcode[height = 1cm]{http://texdoc.net/texmf-dist/doc/fonts/amsfonts/amssymb.pdf}}
	 \item[amscd] Pacote para criar diagramas comutativos\footnote{veja também o 
   pacote tikz:
   \url{http://ctan.math.washington.edu/tex-archive/graphics/pgf/contrib/tikz-cd/tikz-cd-doc.pdf}}\\	
   {\small\url{http://linorg.usp.br/CTAN/macros/latex/required/amsmath/amscd.pdf}}
   \marginpar{\qrcode[height = 1cm]{http://linorg.usp.br/CTAN/macros/latex/required/amsmath/amscd.pdf}}
\end{description}

%------------------------------------------------------------------------------
\subsection*{\texttt{\textbackslash usepackage\{mathtools\}}}
%------------------------------------------------------------------------------
O \emph{Mathtools} fornece muitas ferramentas úteis para a composição
matemática.
Ele é baseado no \texttt{amsmath} e corrige várias deficiências deste e do 
\LaTeX\ padrão.\\
{\small\url{http://linorg.usp.br/CTAN/macros/latex/contrib/mathtools/mathtools.pdf}}
\marginpar{\qrcode[height = 1cm]{http://linorg.usp.br/CTAN/macros/latex/contrib/mathtools/mathtools.pdf}}

%------------------------------------------------------------------------------
\subsection*{\texttt{\textbackslash usepackage\{systeme\}}}
%------------------------------------------------------------------------------
Pacote para construir sistemas de equações.\\
{\small\url{http://linorg.usp.br/CTAN/macros/generic/systeme/systeme_fr.pdf}}
\marginpar{\qrcode[height = 1cm]{http://linorg.usp.br/CTAN/macros/generic/systeme/systeme_fr.pdf}}

%------------------------------------------------------------------------------
\subsection*{\texttt{\textbackslash usepackage\{esint\}}}
%------------------------------------------------------------------------------
Para produzir diversos tipos de integrais.\\
{\small\url{http://linorg.usp.br/CTAN/macros/latex/contrib/esint/esint-doc.pdf}}
\marginpar{\qrcode[height = 1cm]{http://linorg.usp.br/CTAN/macros/latex/contrib/esint/esint-doc.pdf}}

%------------------------------------------------------------------------------
\subsection*{\texttt{\textbackslash usepackage\{array\}}} 
%------------------------------------------------------------------------------
Pacote para construção e modificação de diversos parâmetros numa tabela.
A classe \texttt{ativmatUFRB.cls} está programada para um limite de 20 colunas
numa tabela,  \verb|\setcounter{MaxMatrixCols}{20}|.\\
{\small\url{http://linorg.usp.br/CTAN/macros/latex/required/tools/array.pdf}}
\marginpar{\qrcode[height = 1cm]{http://linorg.usp.br/CTAN/macros/latex/required/tools/array.pdf}}

%------------------------------------------------------------------------------
\subsection*{\texttt{\textbackslash usepackage\{esvect\}}}
%------------------------------------------------------------------------------
Essencial para produzir vetores.\\
{\small\url{http://linorg.usp.br/CTAN/macros/latex/contrib/esvect/esvect.pdf}}
\marginpar{\qrcode[height = 1cm]{http://linorg.usp.br/CTAN/macros/latex/contrib/esvect/esvect.pdf}}

%------------------------------------------------------------------------------
\subsection*{\texttt{\textbackslash usepackage\{graphicx\}}}
%------------------------------------------------------------------------------
Dentre outras funcionalidades, esse pacote é essencial para inserção de figuras
e ambiente para estas num documento.
A classe \texttt{ativmatUFRB.cls} foi coonfigurada para que as figuras sejam 
buscadas na pasta ``figs'' (usou-se o comando \verb|\graphicspath{{./figs/}}|).
Portanto, \textbf{todas} as figuras da lista de atividade devem estar na pasta
intitulada ``figs''.\\
{\small\url{http://linorg.usp.br/CTAN/macros/latex/required/graphics/grfguide.pdf}}
\marginpar{\qrcode[height = 1cm]{http://linorg.usp.br/CTAN/macros/latex/required/graphics/grfguide.pdf}}

%------------------------------------------------------------------------------
\subsection*{\texttt{\textbackslash usepackage[table]\{xcolor\}}}
%------------------------------------------------------------------------------
Produz inúmeras cores\footnote{veja exemplos com \texttt{pstricks}:
\url{http://linorg.usp.br/CTAN/macros/latex/contrib/xcolor/xcolor2.pdf}}
estilizadas.\\ 
{\small\url{http://linorg.usp.br/CTAN/macros/latex/contrib/xcolor/xcolor.pdf}}
\marginpar{\qrcode[height = 1cm]{http://linorg.usp.br/CTAN/macros/latex/contrib/xcolor/xcolor.pdf}}

%------------------------------------------------------------------------------
\subsection*{\texttt{\textbackslash usepackage\{enumerate\}}}
%------------------------------------------------------------------------------
Implementa modificações em ambientes com listas enumeradas.\\
{\small\url{http://linorg.usp.br/CTAN/macros/latex/required/tools/enumerate.pdf}}
\marginpar{\qrcode[height = 1cm]{http://linorg.usp.br/CTAN/macros/latex/required/tools/enumerate.pdf}}

%------------------------------------------------------------------------------
\subsection*{\texttt{\textbackslash usepackage\{fancybox\}}}
%------------------------------------------------------------------------------
Produz caixas estilizadas (ovais, com sombra, etc.).\\
{\small\url{http://linorg.usp.br/CTAN/macros/latex/contrib/fancybox/fancybox-doc.pdf}}
\marginpar{\qrcode[height = 1cm]{http://linorg.usp.br/CTAN/macros/latex/contrib/fancybox/fancybox-doc.pdf}}

%------------------------------------------------------------------------------
\subsection*{\texttt{\textbackslash usepackage\{setspace\}}}
%------------------------------------------------------------------------------
Pacote para espaçamento entre linhas.
A classe \texttt{ativmatUFRB.cls}, no ambiente 
\verb|\begin{Atividade}...\end{Atividade}| já está configurado para espaçamento
entre linhas de 1,5. \\
{\small\url{https://www.ctan.org/pkg/setspace}}
\marginpar{\qrcode[height = 1cm]{https://www.ctan.org/pkg/setspace}}

%------------------------------------------------------------------------------
\subsection*{\texttt{\textbackslash usepackage\{booktabs\}}}
%------------------------------------------------------------------------------
Para tabelas estilizadas e de ótima qualidade tipográfica.\\ 
{\small\url{http://linorg.usp.br/CTAN/macros/latex/contrib/booktabs/booktabs.pdf}}
\marginpar{\qrcode[height = 1cm]{http://linorg.usp.br/CTAN/macros/latex/contrib/booktabs/booktabs.pdf}}

%------------------------------------------------------------------------------
\subsection*{\texttt{\textbackslash usepackage\{multicol\}}}
%------------------------------------------------------------------------------
Para escrever texto em várias colunas (máximo de 10).\\
{\small\url{http://linorg.usp.br/CTAN/macros/latex/required/tools/multicol.pdf}}
\marginpar{\qrcode[height = 1cm]{http://linorg.usp.br/CTAN/macros/latex/required/tools/multicol.pdf}}

%------------------------------------------------------------------------------
\subsection*{\texttt{\textbackslash usepackage[labelfont=bf,font=small]\{caption\}}}
%------------------------------------------------------------------------------
Modifica legendas em ambientes flutuantes (tabelas e figuras).
A presente classe está configurada para que a fonte da legenda esteja em um 
tamanho menor do que a fonte do corpo do texto; e, a marcação de cada legenda 
está em negrito.\\
{\small\url{http://linorg.usp.br/CTAN/macros/latex/contrib/caption/caption-eng.pdf}}
\marginpar{\qrcode[height = 1cm]{http://linorg.usp.br/CTAN/macros/latex/contrib/caption/caption-eng.pdf}}

%------------------------------------------------------------------------------ 
\subsection*{\texttt{\textbackslash usepackage\{hyperref\}}}
%------------------------------------------------------------------------------
O pacote \emph{hiperref} é usado para manipular comandos de referência cruzada
no \LaTeX\ e para produzir links de hipertexto no documento.
Na presente classe os \emph{links} estão ativados para cor azul.\\
{\small\url{http://linorg.usp.br/CTAN/macros/latex/contrib/hyperref/doc/manual.pdf}}
\marginpar{\qrcode[height = 1cm]{http://linorg.usp.br/CTAN/macros/latex/contrib/hyperref/doc/manual.pdf}}

%------------------------------------------------------------------------------ 
\subsection*{\texttt{\textbackslash usepackage\{units\}}}
%------------------------------------------------------------------------------
Para padronizar unidades.\\
{\small\url{https://ctan.dcc.uchile.cl/macros/latex/contrib/units/units.pdf}}
\marginpar{\qrcode[height = 1cm]{https://ctan.dcc.uchile.cl/macros/latex/contrib/units/units.pdf}}
