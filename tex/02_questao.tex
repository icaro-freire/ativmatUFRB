%==============================================================================
% Questão 02 
%------------------------------------------------------------------------------
\questao Uma sequência $ (x_n)_{_{n\in\mathbb{N}}} $ é dita \textit{Sequência de Cauchy} 
quando:
\begin{center}
  \fbox
  {
    \begin{minipage}{.7\textwidth}
      \sffamily
      Dado arbitrariamente um número real $ \varepsilon > 0 $, 
      pode-se obter $ n_{0}\in \mathbb{N} $ tal que:
      \[
        m > n_{0} \quad \textrm{e} \quad n > n_{0} 
        \Longrightarrow
        \left|x_{m} - x_{n}\right| < \varepsilon.
      \]
    \end{minipage}
  }
\end{center}
\begin{itens}
  \item Mostre que toda sequência convergente é de Cauchy.
\end{itens}

%==============================================================================
