%==============================================================================
% Questão 09
%------------------------------------------------------------------------------

\questao Cinco corredores foram examinados para determinar a quantidade máxima 
de aspiração de oxigênio, que é uma medida usada para caracterizar a situação 
cardiovascular de uma pessoa.
Os resultados estão na Tabela~\ref{tab:corredor}, onde ``$x$'' é o número de 
segundos no melhor tempo feito em um quilômetro e ``$y$'' é o número de 
mililitros por minuto, por quilograma de peso corporal da aspiração máxima de 
oxigênio do corredor.

\begin{table}[!htbp] %----------------------------------------------------------> Ambiente para Tabela/ Posicionamento no texto.
 \caption{Segundos por melhor corredor} %---------------------------------------> Legendas da tabela.
 \label{tab:corredor} %---------------------------------------------------------> Marcação para referencias cruzadas.
 \centering %-------------------------------------------------------------------> Centraliza a tabela.
 \begin{tabular}{l c c c c c} %---------------------> Tabela estilizada/ Cada letra é uma coluna (c: centralizada; l: à esquerda; r: à direita).
  \toprule %------> Linha superior.
                & \textbf{Corredor A} & \textbf{Corredor B} & \textbf{Corredor C} & \textbf{Corredor D} & \textbf{Corredor E}\\
  \midrule %------> Linha média.
   $\mathbf{x}$ & 300,5               & 350,6               & 407,3               & 326,2               & 512,8\\
   $\mathbf{y}$ & 350,2               & 325,8               & 375,6               & 418,5               & 400,2\\
  \bottomrule %---> Linha inferior.
 \end{tabular}
\end{table}
	
\begin{itens}
  \item Trace o diagrama de dispersão.
  \item Ache a reta de regressão para os dados da tabela.
  \item Use a reta de regressão para estimar a máxima aspiração de oxigênio de 
        um corredor, cujo melhor tempo em uma milha é de \unit{340,4.s}. %------> `\unit{}` padroniza unid. de media. Note o 'ponto' para separação.
\end{itens}
%==============================================================================