%==============================================================================
% Questão 01
%------------------------------------------------------------------------------
%
% Note que as questões são iniciadas com \questao. ¯\_(ツ)_/¯
%
\questao Na Figura~\ref{fig:espiral} temos uma espiral\footnote{%
  \urlb{https://en.wikipedia.org/wiki/Spiral} %---------------------------------> \urlb{} deixa a url em negrito (caso contrário use \url{})
}
formada por semicírculos cujos centros pertencem ao eixos das abscissas. 
Se o raio do primeiro semicírculo é igual a $ 1 $ e o raio de cada semicírculo é 
igual a metade do semicírculo anterior, determine:

\begin{figure}[!htbp] %---------------------------------------------------------> Ambiente para figuras.
  \centering %------------------------------------------------------------------> Centralizando a figura.
  \includegraphics[width = 0.35 \textwidth]{espiral} %--------------------------> Comando para inserir figura.
  \caption{Um tipo de espiral} %------------------------------------------------> Legenda da figura.
  \label{fig:espiral} %---------------------------------------------------------> Marcação interna para referências cruzadas.
\end{figure}

\begin{itens} %-----------------------------------------------------------------> Ambiente para listas.
  \item o comprimento da espiral. \Resp{$2\pi$} %-------------------------------> Comando para as respostas.
  \item a abscissa do ponto assintótico da espiral. \Resp{$4/3$}
\end{itens}
%==============================================================================